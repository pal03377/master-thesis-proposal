% Add common preamble to the document
\input{common/preamble.tex}
\def\proposal{Proposal for}

%%%%%%%%%%%%%%%%%%%%%%%%%%%%%%%%%%%%%%%%%%%%%%%%%%%%%%%%%%%
% Theses specific packages go here
%%%%%%%%%%%%%%%%%%%%%%%%%%%%%%%%%%%%%%%%%%%%%%%%%%%%%%%%%%%
\usepackage[nolist]{acronym}

%%%%%%%%%%%%%%%%%%%%%%%%%%%%%%%%%%%%%%%%%%%%%%%%%%%%%%%%%%%
% Begin of document
%%%%%%%%%%%%%%%%%%%%%%%%%%%%%%%%%%%%%%%%%%%%%%%%%%%%%%%%%%%
\begin{document}
\setlength{\evensidemargin}{22pt}
\setlength{\oddsidemargin}{22pt}


\hypersetup{pdfborder={0 0 0}, pdfauthor={\author}, pdftitle={\title}}

\lstset{showspaces=false, numbers=left, frame=single, basicstyle=\small}

%------- Title setup -------
\include{common/titlepage}

\selectlanguage{english}
\pagenumbering{arabic}

\fancyhead{}
\pagestyle{fancy}
\fancyhead[LE]{\slshape \leftmark}
\fancyhead[RO]{\slshape \rightmark}
\headheight=15pt

%------- Start of Proposal -------
\section*{Introduction}

        \missing{ Introduction
                \begin{itemize}
                        \item Introduce the reader to the general setting
                        \item What is the environment?
                        \item What are the tools in use?
                \end{itemize}
        }
Intro and Problem:
\begin{itemize}
        \item Number of students in Informatics is increasing
        \item Exercises are important
        \item in large courses it is hard to give feedback individually
        \item There are courses with automatically tested exercises
        \item Those tests are difficult to create and get right - students often complain about some of them
        \item Manually graded exercises are more flexible, but it is hard to give feedback to all students
        \item There exists a system for automatically generating feedback on text and modeling exercises called Athene, which is integrated into the LMS Artemis
        \item This system however cannot provide feedback on programming exercises, and it is currently hard to extend as it only supports a single approach to each of the following steps performed:
        \begin{itemize}
                \item Segmenting the exercise into parts
                \item Embedding the parts to a vector representation
                \item Clustering the vector representations
                \item Choosing feedback from the same cluster (which currently is not even done within Athene)
        \end{itemize}
\end{itemize}
The number of students in Computer Science courses at universities is increasing steadily. This fact also means that the number of exercise submissions in those courses has increased drastically.
These circumstances pose a problem for the instructors, as it is hard to give feedback to all students individually. There exist courses using automatically tested exercises, but those are often difficult to create and get right. Manually graded exercises can provide more individuality, but it is hard to give feedback to all students.

Artemis is an online learning platform for exercise management. It is being developed at the Technical University of Munich and used by several other universities in Germany. Athene, a system for automatically generating feedback on text and modeling exercises, is integrated into Artemis. It works by essentially clustering submissions into groups using machine learning techniques and then choosing feedback from the same cluster. 
Athene cannot provide feedback on programming exercises, and it is currently hard to extend as it only supports one approach to each of the steps performed in the feedback generation process.

\section*{Problem}
        \missing{ Problem description
                \begin{itemize}
                        \item What is/are the problem(s)? 
                        \item Identify the actors and use these to describe how the problem negatively influences them.
                        \item Do not present solutions or alternatives yet!
                        \item Present the negative consequences in detail 
                \end{itemize}
        }
        
\section*{Motivation}
        \missing{ Thesis Motivation
                \begin{itemize}
                        \item Outline why it is important to solve the problem
                        \item Again use the actors to present your solution, but don't be to specific
                        \item Be visionary! 
                        \item If applicable, motivate with existing research, previous work 
                \end{itemize}
        }
        
\section*{Objective}
        \missing{ Thesis Objective
                \begin{itemize}
                        \item What are the main goals of your thesis?
                \end{itemize}
        }
        
\section*{Schedule}
        \missing{ Thesis Schedule
                \begin{itemize}
                        \item When will the thesis Start (Always 15th of Month) 
                        \item Create a rough plan for your thesis (separate the time in sprints with a length of 2-4 Weeks)
                        \item Each sprint should contain several work items - Again keep it high-level and make to keep your plan realistic
                        \item Make sure the work-items are measurable and deliverable 
                        \item No writing related tasks! (e.g. "Write Analysis Chapter")
                \end{itemize}
        }



\clearpage
\include{common/acronyms}
\clearpage
\bibliography{thesis}
\bibliographystyle{alpha}

\end{document}
