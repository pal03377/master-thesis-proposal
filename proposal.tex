% Add common preamble to the document
%% This file is shared between thesis and proposal

\documentclass[a4paper,12pt,twoside]{report}

\usepackage[scaled]{helvet}
\usepackage{url}
\usepackage{cite}
\usepackage{listings}
\usepackage[pdftex]{graphicx}
\usepackage[hang,small,bf]{caption}
\usepackage{styles/tum}
\usepackage{setspace}
\usepackage[german,english]{babel}
\usepackage{float}
\usepackage{floatflt}
\usepackage{fancyhdr}
\usepackage{color}
\usepackage{booktabs}
\usepackage[pdftex,bookmarks=true,plainpages=false,pdfpagelabels=true]{hyperref}	%TODO make yourself familiar with \label, \ref and \hyperref for referencing figures, tables, chapters, etc.
\usepackage{mdwlist}
\usepackage{enumerate}
\usepackage{array}
\usepackage{longtable}
\usepackage[utf8]{inputenc}
\usepackage[capitalize, noabbrev]{cleveref}
\usepackage{wasysym}
\usepackage{todonotes}

% Path for graphics
\graphicspath{{figures/}}

% Include the Thesis metadata like title, author, etc. 
\input{metadata}

%%%%%%%%%%%%%%%%%%%%%%%%%%%%%%%%%%%%%%%%%%%%%%%%%%%%%%%%%%%%%%%%%%%%%%%%%%%%%%%%%%%%%%%%%%%%%%%%%
% Custom Commands for this template
%%%%%%%%%%%%%%%%%%%%%%%%%%%%%%%%%%%%%%%%%%%%%%%%%%%%%%%%%%%%%%%%%%%%%%%%%%%%%%%%%%%%%%%%%%%%%%%%%

% Annotate feedback you received 
\newcommand{\feedback}[1]{\todo[inline,color=green,caption={}]{#1}}

% State what is missing in this spot
\newcommand{\missing}[1]{\todo[inline,color=yellow,caption={}]{#1}}

% Inline to do note: 
\newcommand{\TODO}[1]{\todo[inline,caption={}]{#1}}





\def\proposal{Proposal for}

%%%%%%%%%%%%%%%%%%%%%%%%%%%%%%%%%%%%%%%%%%%%%%%%%%%%%%%%%%%
% Theses specific packages go here
%%%%%%%%%%%%%%%%%%%%%%%%%%%%%%%%%%%%%%%%%%%%%%%%%%%%%%%%%%%
\usepackage[nolist]{acronym}
\usepackage{csquotes}

%%%%%%%%%%%%%%%%%%%%%%%%%%%%%%%%%%%%%%%%%%%%%%%%%%%%%%%%%%%
% Begin of document
%%%%%%%%%%%%%%%%%%%%%%%%%%%%%%%%%%%%%%%%%%%%%%%%%%%%%%%%%%%
\begin{document}
\setlength{\evensidemargin}{22pt}
\setlength{\oddsidemargin}{22pt}


\hypersetup{pdfborder={0 0 0}, pdfauthor={\author}, pdftitle={\title}}

\lstset{showspaces=false, numbers=left, frame=single, basicstyle=\small}

%------- Title setup -------
\thispagestyle{empty}
{
\sffamily

\vspace{1cm}
\begin{center}
\oTUM{4cm}

\vspace{5mm}     
{\LARGE \bf \sffamily Technical University of Munich}

\vspace{5mm}
{\Large School of Computation, Information and Technology \\ -- Informatics -- }	
\vspace{1mm}
\end{center}

\vspace{15mm}

\begin{center}
        {\large {\proposal} {\degree}'s Thesis in \program}
\vspace{8mm}

\begin{spacing}{1.3}
{\LARGE \bf \sffamily \title}\\
\vspace{8mm}

{\LARGE \titleGer}\\
\vspace{8mm}
\end{spacing}

\begin{tabular}{ll}
\large Author:           & \large \author     \\[2mm]
\large Supervisor:       & \large \supervisor \\[2mm]				
\large Advisor:	         & \large \advisor    \\[2mm]
\ifx\proposal\empty\else
\large Start Date:       & \large \startdate  \\[2mm]
\fi
\large Submission Date:  & \large \date
\end{tabular}

\end{center}
}


\selectlanguage{english}
\pagenumbering{arabic}

\fancyhead{}
\pagestyle{fancy}
\fancyhead[LE]{\slshape \leftmark}
\fancyhead[RO]{\slshape \rightmark}
\headheight=15pt

% For wording consistency:
% An "assessment" is a collection of feedbacks + credit scores
% A "feedback" is a single feedback without a score
% A "suggestion" is an assessment that is suggested to a tutor
% A "submission" is a single submission of a student

%------- Start of Proposal -------
\section*{Introduction}
% - Introduce the reader to the general setting
% - What is the environment?
% - What are the tools in use?

% TODO: Briefly explain words from above within the text

The number of students in computer science courses at universities is steadily increasing. At the Technical University of Munich, the number of full-time students\footnote{i.e., full-time equivalents} has recently increased by more than 2,400 within five years~\footnote{TUM in Zahlen 2020, \url{https://mediatum.ub.tum.de/doc/1638190/1638190.pdf}}.
These circumstances present a challenge for instructors, as providing feedback to all students individually can be difficult. Manually graded exercises can provide more individuality, but it is a difficult task to assess all submissions in a timely manner.

Artemis is an online learning platform for exercise management supporting automatic code testing for grading~\cite{ArTEMiS}. The Research Group for Applied Software Engineering at the Technical University of Munich is the main contributor to Artemis, but the system is also in use at several other universities.

Athene, a system for (semi-)automated assessment of text exercises\footnote{\url{https://github.com/ls1intum/Athena}}, is integrated into Artemis~\cite{cofee}. Athene is a microservice-based system written in Python that generates \textit{suggestions} for \textit{assessments} of text exercise \textit{submissions}.


\section*{Problem}
% - What is/are the problem(s)? 
% - Identify the actors and use these to describe how the problem negatively influences them.
% - Do not present solutions or alternatives yet!
% - Present the negative consequences in detail

Currently, Athene is bound to one approach for each step in the assessment generation process. This decreases the flexibility and extensibility of the system and is one of the main points we will improve in this thesis.
On a more practical level, Athene does not support programming exercises, which are a common type of exercise in Computer Science courses. Support for programming exercises is one of the main advantages of Artemis over other exercise management systems (such as Moodle\footnote{\url{https://moodle.org}}), so it would be beneficial to extend Athene to support programming exercises as well.

Two types of actors could have problems with the current status of Athene:
\begin{itemize}
    \item \textbf{Tutors} in courses with manually graded programming exercises: They cannot profit from Athene's assessment generation capabilities, as Athene does not support programming exercises. This means that they won't get any automatically generated suggestions for programming exercises, which would save them a lot of time.
    For textual exercises, Athene currently provides suggestions for around 45\% of the submissions~\cite{cofee2}. In the future, further work might improve this number even further, based on the generalizations of this thesis that allow the system to support different approaches for each step in the assessment generation process.
    \item It is difficult for \textbf{developers} of Athene to integrate additional approaches and features into Athene, as the system is currently bound to one approach for each step in the generation process. Artemis also chooses the assessment suggestions (outside of Athene), which makes it impossible to change the suggestions independently of Artemis.
\end{itemize}

\section*{Motivation}
%- Outline why it is important to solve the problem
% - Again use the actors to present your solution, but don't be to specific
% - Be visionary! 
% - If applicable, motivate with existing research, previous work 

As student numbers will continue to rise (especially in Computer Science courses), the number of assessments which tutors have to make will increase as well. Extending Athene to support programming exercises would allow tutors to profit from its assessment generation capabilities and save them a lot of time.

Generalizing each of the steps in the process would also allow developers to integrate additional approaches and features into Athene more easily.
For example, we could replace the clustering step with a different clustering algorithm, which might improve the quality of the suggestions. Past work like an evaluation of the helpfulness of assessments from Athene~\cite{atheneTracking} or efforts to make it more language-independent~\cite{atheneLanguage} could also potentially have profited from more generalization.

Furthermore, groundwork on a more flexible system might enable applying relatively new innovations like ChatGPT\footnote{\url{https://openai.com/blog/chatgpt}} to Athene, if the suggestion generation does not only support choosing existing assessments, but also generating new ones.

\section*{Objective}
% - What are the main goals of your thesis?
In the thesis, we want to enable Athene to support programming exercises and to generalize the steps in the assessment generation process. We will do this in a way that does not affect the current functionality of Athene and does not introduce any new bugs.

% One of the advantages of having the possibility to choose dynamically between different ways of performing the steps in the assessment generation process is that it would allow us to switch them as necessary to implement suggestions for programming exercises as well. 

\subsection*{Generalization of the Assessment Generation Process}
By moving the actual choice of assessments from Artemis to Athene, we want to make it possible to change the suggestions independently of Artemis. Also, Athene should offer a more general API: Currently, it computes the submission clusters and sends them to Artemis in a ProtoBuf format. A fixed interface contract for the API would make it easier to integrate Athene into other systems in the future.

\subsection*{Programming Exercise Assessment Generation}
We will extend Athene to support programming exercises. This will involve extensive analysis of the current state of the art in automatic assessment generation for programming exercises, as well as the development of new approaches for the steps in the generation process. Additionally, we will deploy and test the new system to ensure its reliability and effectiveness.

\subsection*{Adding Programming Exercise Suggestions to Artemis}
Because Artemis currently does not support automatic programming exercise suggestions, we also want to make those available in Artemis. Tutors should be able to accept or reject the suggestions in the web interface, similar to the existing automatic assessments on text exercises, which Athene generates. 

\subsection*{Adding Programming Exercise Suggestions to Themis}

Themis\footnote{\url{https://github.com/ls1intum/Themis}} is a new iPad app, which supports grading of programming exercises. The app was a project in the practical course \enquote{iPraktikum} at the Technical University of Munich.
Themis will also be able to use the programming exercise suggestions from Athene. It should do so by only communicating with Artemis directly, which means that Themis will not need to communicate with Athene directly. This decreases the coupling between Themis and Athene and makes it easier to change the generation system in the future.

\subsection*{Iterating on Existing Work from ThemisML}
Currently, Themis supports automatic suggestions for programming exercises by using its own assessment generation system prototype, ThemisML\footnote{\url{https://github.com/ls1intum/Themis-ML}}.
Parts of this system will be useful for integrating programming exercise suggestions into Athene. ThemisML uses codeBERT~\cite{codeBERT} to compute pairwise similarity scores between submissions to suggest appropriate suggestions. We want to move this functionality into Athene to completely replace the current generation system in ThemisML.

\subsection*{Utilizing codeBERT for Assessment Generation}
We will evaluate a variety of approaches for effectively segmenting, embedding and clustering programming exercises to improve on the efficiency and quality of the suggestions and to integrate them into Athene.
One of the main approaches we will evaluate is codeBERT~\cite{codeBERT}, which is a BERT model that was trained on code. Currently it is used in ThemisML to compute pairwise similarity scores between submissions. We will evaluate how it can be used in a more general assessment generation system.


\section*{Schedule}
% - When will the thesis Start (Always 15th of Month) 
% - Create a rough plan for your thesis (separate the time in sprints with a length of 2-4 Weeks)
% - Each sprint should contain several work items - Again keep it high-level and make to keep your plan realistic
% - Make sure the work-items are measurable and deliverable 
% - No writing related tasks! (e.g. "Write Analysis Chapter")

The following schedule roughly outlines the work items for the thesis. It will start on March 15th, 2023 and last for 6 months, roughly 26 weeks.

\begin{itemize}
    \item \textit{1 week:} \textbf{Familiarize} with the existing code base of Athene and Artemis, especially the assessment choice within Artemis.
    \item \textit{4 weeks:} \textbf{Generalize} the existing microservices for segmentation, embedding and clustering in Athene: It should be possible to easily switch between internal modules.
    \item \textit{6 weeks:} \textbf{Move the actual choice of assessments} from Artemis to a new microservice within Athene: It should offer more general API endpoints for storing submissions, for storing assessments and for providing suggestions. Artemis should use these endpoints instead of the current ones.
    \begin{itemize}
        \item \textit{3 weeks:} \textbf{Re-implement} the new microservice in Athene. It should be possible to switch to any module for the assessment choice as for the other microservices as before. The selection should behave like it currently does in Artemis, but we want to implement it in Python instead of Java.
        \item \textit{2 weeks:} \textbf{Replace} the current assessment choice in Artemis with the new microservice.
        \item \textit{1 week:} \textbf{Test} the new microservice and the changes in Artemis. Ensure that the changes do not affect the functionality of Artemis.
    \end{itemize}
    \item \textit{Within the first 11 weeks:} \textbf{Evaluate} different possibilities for segmenting, embedding and clustering programming exercises. % Maybe this should be first?
    \item \textit{6 weeks:} Add support for \textbf{programming exercises} in Athene:
    \begin{itemize}
        \item \textit{4 weeks:} Add support for programming exercise segmentation, embedding and clustering as well as suggestion generation to Athene - based on prior research.
        \item \textit{1 week:} Implement communication between Artemis and Athene for programming exercises into Artemis.
        \item \textit{1 week:} Add support for displaying automatic suggestions for programming exercises in Artemis.
    \end{itemize}
    \item \textit{2 weeks:} \textbf{Integrate} suggestions \textbf{into the Themis iPad app}: It should be able to display the programming exercise suggestions from Athene.
    \item \textit{3 weeks:} Implement \textbf{improvements and bug fixes} still necessary for a stable release of Athene.
    \item \textit{4 weeks:} Write \textbf{documentation and the (remaining) thesis} for the project.
\end{itemize}


\clearpage
\begin{acronym}
\acro{GUI}{Graphical User Interface}
\end{acronym}

\clearpage
\bibliography{thesis}
\bibliographystyle{alpha}

\end{document}
