% Add common preamble to the document
\input{common/preamble.tex}
\def\proposal{Proposal for}

%%%%%%%%%%%%%%%%%%%%%%%%%%%%%%%%%%%%%%%%%%%%%%%%%%%%%%%%%%%
% Theses specific packages go here
%%%%%%%%%%%%%%%%%%%%%%%%%%%%%%%%%%%%%%%%%%%%%%%%%%%%%%%%%%%
\usepackage[nolist]{acronym}

%%%%%%%%%%%%%%%%%%%%%%%%%%%%%%%%%%%%%%%%%%%%%%%%%%%%%%%%%%%
% Begin of document
%%%%%%%%%%%%%%%%%%%%%%%%%%%%%%%%%%%%%%%%%%%%%%%%%%%%%%%%%%%
\begin{document}
\setlength{\evensidemargin}{22pt}
\setlength{\oddsidemargin}{22pt}


\hypersetup{pdfborder={0 0 0}, pdfauthor={\author}, pdftitle={\title}}

\lstset{showspaces=false, numbers=left, frame=single, basicstyle=\small}

%------- Title setup -------
\include{common/titlepage}

\selectlanguage{english}
\pagenumbering{arabic}

\fancyhead{}
\pagestyle{fancy}
\fancyhead[LE]{\slshape \leftmark}
\fancyhead[RO]{\slshape \rightmark}
\headheight=15pt

%------- Start of Proposal -------
\section*{Introduction}
% - Introduce the reader to the general setting
% - What is the environment?
% - What are the tools in use?

The number of students in Computer Science courses at universities is increasing steadily. This fact also means that the number of exercise submissions in those courses has increased drastically.
These circumstances pose a problem for the instructors, as it is hard to give feedback to all students individually. There exist courses using automatically tested exercises, but those are often difficult to create and get right. Manually graded exercises can provide more individuality, but it is hard to give feedback to all students.

Artemis is an online learning platform for exercise management. It is being developed at the Technical University of Munich and used by several other universities in Germany. 
Athene, a system for automatically generating feedback on text exercises, is integrated into Artemis. It works by essentially clustering submissions into groups using machine learning techniques and then choosing feedback from the same cluster. The actual steps in the feedback generation process are in detail~\cite{cofee}:
\begin{itemize}
    \item \textbf{Segmentation} of the submission: Athene splits the submission into parts by topics, which are identified using the presence or absence of certain keywords.
    \item \textbf{Language Embedding} of the parts: Athene embeds the parts to a vector representation using a pre-trained ELMo language model~\cite{deepContextualizedWordRepresentations}.
    \item \textbf{Clustering} of the vector representations: Athene clusters the vector representations using the Hierarchical Density-Based Spatial Clustering algorithm~\cite{hdbsc}.
\end{itemize}

Each of these steps is currently performed by one microservice, all of which are written in Python and implement one approach to each of the steps.
In addition, there is a tracking service for logging the feedback generation process implemented by Petry~\cite{atheneTracking} and a load balancer for distributing the load between the microservices~\cite{atheneLoadBalancer}.
Currently, Athene sends a list of clusters and a list of segments to Artemis. Artemis then chooses feedbacks from the same cluster for the suggestions.


\section*{Problem}
% - What is/are the problem(s)? 
% - Identify the actors and use these to describe how the problem negatively influences them.
% - Do not present solutions or alternatives yet!
% - Present the negative consequences in detail

Currently, Athene is bound to one approach for each step in the feedback generation process. This decreases the flexibility and extensibility of the system and should be improved.
On a more practical level, Athene does not support programming exercises, which are a common type of exercise in Computer Science courses.

Two types of actors could have problems with the current status of Athene:
\begin{itemize}
    \item \textbf{Tutors} in courses with manually graded programming exercises: They cannot profit from Athene's feedback generation capabilities, as Athene does not support programming exercises. This means that they won't get any automatically generated feedback suggestions for programming exercises, which would save them a lot of time.
    For textual exercises, Athene currently provides feedback suggestions for around 45\% of the submissions~\cite{cofee2}. This number might be improved even further in the future if the system is generalized to support different approaches for each step in the feedback generation process.
    \item It is difficult for \textbf{Developers} of Athene to integrate additional approaches and features into Athene, as the system is currently bound to one approach for each step in the feedback generation process. The choice of the feedback suggestion is also done by Artemis (outside of Athene), which makes it impossible to change the feedback suggestion independently of Artemis.
\end{itemize}

\section*{Motivation}
%- Outline why it is important to solve the problem
% - Again use the actors to present your solution, but don't be to specific
% - Be visionary! 
% - If applicable, motivate with existing research, previous work 

As student numbers will continue to rise (especially in Computer Science courses), the number of feedbacks which tutors have to give will increase as well. Extending Athene to support programming exercises would allow tutors to profit from its feedback generation capabilities and save them a lot of time.

Generalizing each of the steps in the feedback generation process would also allow developers to integrate additional approaches and features into Athene more easily.
For example, the clustering step could be replaced by a different clustering algorithm, which might improve the quality of the feedback suggestions. Past work like an evaluation of the helpfulness of feedbacks given by Athene by Petry~\cite{atheneTracking} or efforts to make it more language-independent~\cite{atheneLanguage} could also potentially have profited from more generalization.

Furthermore, groundwork on a more flexible system might enable relatively new innovations like ChatGPT to be used in Athene, if the feedback suggestion generation not only supports choosing existing feedbacks, but also generating new ones.

\section*{Objective}
        \missing{ Thesis Objective
                \begin{itemize}
                        \item What are the main goals of your thesis?
                \end{itemize}
        }
        
\section*{Schedule}
        \missing{ Thesis Schedule
                \begin{itemize}
                        \item When will the thesis Start (Always 15th of Month) 
                        \item Create a rough plan for your thesis (separate the time in sprints with a length of 2-4 Weeks)
                        \item Each sprint should contain several work items - Again keep it high-level and make to keep your plan realistic
                        \item Make sure the work-items are measurable and deliverable 
                        \item No writing related tasks! (e.g. "Write Analysis Chapter")
                \end{itemize}
        }



\clearpage
\include{common/acronyms}
\clearpage
\bibliography{thesis}
\bibliographystyle{alpha}

\end{document}
